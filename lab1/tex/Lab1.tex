\documentclass[12pt,a4paper]{article}

\usepackage[utf8]{inputenc}
\usepackage[T2A]{fontenc}
\usepackage[russian]{babel}

\usepackage{float}
\usepackage{amsmath, amssymb}
\usepackage{graphicx}
\usepackage{geometry}
\usepackage{hyperref}

\usepackage{alltt}

\geometry{margin=2.5cm}

\setlength{\parindent}{0pt}
\setlength{\parskip}{0.8em}

\hypersetup{
    pdfborderstyle={/S/U/W 1},	
    colorlinks=true,
    linkcolor=blue,
    filecolor=magenta,      
    urlcolor=blue,
    pdfpagemode=FullScreen,
    }

\begin{document}

\setcounter{equation}{0}

\section*{Калибровка камеры по плоскому шаблону}

\subsection*{Введение}

Калибровка камеры позволяет восстановить количественную связь между трёхмерными координатами точек сцены и их двумерными проекциями на изображении. Эта связь определяется параметрами камеры, включающими внутренние характеристики (фокусное расстояние, положение главной точки), а также внешние параметры, описывающие положение камеры в пространстве.

В данной лабораторной работе рассматривается классический метод калибровки по плоскому шаблону. В основе метода лежит оценка гомографий между плоскостью шаблона и изображением, после чего внутренние параметры камеры восстанавливаются с использованием линейных ограничений, предложенных Чжаном.

\subsection*{\href{https://docs.opencv.org/4.x/d9/d0c/group__calib3d.html}{Геометрическая модель камеры}}

В pinhole-модели проекция трёхмерной точки сцены на изображение описывается уравнением
\begin{equation}
\lambda
\begin{bmatrix}
u \\ v \\ 1
\end{bmatrix}
=
K
\left[
R \mid t
\right]
\begin{bmatrix}
X \\ Y \\ Z \\ 1
\end{bmatrix}.
\label{eq:pinhole}
\end{equation}

Матрица внутренних параметров камеры имеет вид
\begin{equation}
K =
\begin{bmatrix}
f_x & 0 & c_x \\
0 & f_y & c_y \\
0 & 0 & 1
\end{bmatrix}.
\label{eq:K}
\end{equation}


Здесь $f_x$ и $f_y$ — фокусные расстояния в пикселях, а $(c_x, c_y)$ — координаты главной точки изображения. Параметр сдвига осей (skew) считается равным нулю.

Матрица $R \in SO(3)$ и вектор $t \in \mathbb{R}^3$ описывают ориентацию и положение камеры относительно мировой системы координат. В программных библиотеках вращение часто представляется в виде вектора Родрига, однозначно соответствующего матрице $R$.

\bigskip

\begin{figure}[H]
  \centering
  \includegraphics[scale=0.4]{images/Pinhole-Camera-Model.png}
  \caption{геометрия pinhole-модели и системы координат.}
  \label{fig:pinhole1}
\end{figure}

\bigskip

\subsection*{Модель искажений}

Пусть $(x, y)$ — нормализованные координаты точки в системе координат камеры:
\begin{equation}
x = \frac{X_c}{Z_c}, \quad y = \frac{Y_c}{Z_c}.
\end{equation}

Радиальная дисторсия описывается выражением
\begin{equation}
x_r = x \left( 1 + k_1 r^2 + k_2 r^4 + k_3 r^6 \right),
\end{equation}
\begin{equation}
y_r = y \left( 1 + k_1 r^2 + k_2 r^4 + k_3 r^6 \right),
\end{equation}
где
\begin{equation}
r^2 = x^2 + y^2.
\end{equation}

Тангенциальная дисторсия имеет вид
\begin{equation}
x_d = x_r + 2p_1xy + p_2(r^2 + 2x^2),
\end{equation}
\begin{equation}
y_d = y_r + p_1(r^2 + 2y^2) + 2p_2xy.
\end{equation}

Данные параметры описывают отклонения реальной оптической системы от идеальной модели и особенно заметны на периферии изображения.

\begin{figure}[H]
  \centering
  \includegraphics[scale=0.6]{images/Distortion.png}
  \caption{визуализация радиальной и тангенциальной дисторсии.}
  \label{fig:distortion}
\end{figure}

\bigskip

\subsection*{\href{https://docs.opencv.org/4.x/dc/dbb/tutorial_py_calibration.html}{Калибровка камеры средствами OpenCV}}

Алгоритмы калибровки камеры основаны на минимизации ошибки репроекции. 
Под оператором проекции
\[
\pi(K, R, t, X)
\]
понимается отображение трёхмерной точки $X = (X,Y,Z)$ в двумерную точку изображения
с использованием уравнения~\eqref{eq:pinhole} и модели дисторсии.

Формально оператор $\pi(\cdot)$ включает следующие этапы:
\begin{itemize}
\item преобразование точки из мировой системы координат в систему координат камеры;
\item перспективное деление для получения нормализованных координат;
\item применение модели радиальной и тангенциальной дисторсии;
\item переход к пиксельным координатам с использованием матрицы $K$.
\end{itemize}

Ошибка репроекции определяется как
\begin{equation}
e = \frac{1}{N} \sum_{i=1}^{N}
\left\|
\begin{bmatrix}
u_i \\ v_i
\end{bmatrix}
-
\pi(K, R, t, X_i)
\right\|,
\label{eq:reprojection}
\end{equation}
где $(u_i, v_i)$ — наблюдаемые координаты точки на изображении, $\pi(K, R, t, X_i)$ — координаты той же точки, полученные путём проекции.




\subsection*{\href{https://www.youtube.com/watch?v=3NcQbZu6xt8}{Гомография и метод DLT}}

Если все точки шаблона лежат в плоскости $Z = 0$, проекционное уравнение~\eqref{eq:pinhole}
упрощается и принимает вид
\begin{equation}
\lambda
\begin{bmatrix}
u \\ v \\ 1
\end{bmatrix}
=
H
\begin{bmatrix}
X \\ Y \\ 1
\end{bmatrix},
\label{eq:homography}
\end{equation}
где:
\begin{itemize}
\item $(X, Y)$ — координаты точки на плоскости шаблона;
\item $(u, v)$ — координаты соответствующей точки на изображении;
\item $H$ — матрица гомографии размера $3 \times 3$, определяемая с точностью до масштаба.
\end{itemize}

Матрица гомографии записывается в виде
\[
H =
\begin{bmatrix}
h_1 & h_2 & h_3 \\
h_4 & h_5 & h_6 \\
h_7 & h_8 & h_9
\end{bmatrix}.
\]

Каждое соответствие $(X, Y) \leftrightarrow (u, v)$ задаёт два линейных уравнения
относительно элементов $H$. Объединяя уравнения для всех точек, получается
однородная линейная система
\begin{equation}
A h = 0,
\label{eq:dlt}
\end{equation}
где
\[
h =
\begin{bmatrix}
h_1 & h_2 & h_3 & h_4 & h_5 & h_6 & h_7 & h_8 & h_9
\end{bmatrix}^\top
\]
— вектор, составленный из элементов матрицы $H$ построчно.

Матрица $A$ имеет размер $2N \times 9$, где $N$ — число соответствий точек.
Поскольку система~\eqref{eq:dlt} является однородной, ненулевое решение
существует только с точностью до масштабного множителя.

Решение системы~\eqref{eq:dlt} определяется как правый сингулярный вектор матрицы $A$,
соответствующий наименьшему сингулярному значению. Для повышения численной устойчивости
перед построением матрицы $A$ выполняется нормализация координат точек.



\subsection*{\href{https://sangillee.com/2025-07-27-calibrating-cameras-zhang-method/}{Метод Чжана}}

Пусть для каждого изображения плоского шаблона вычислена гомография $H_i$.
Связь гомографии с параметрами камеры задаётся соотношением
\begin{equation}
H_i = K [ r_{i1} \; r_{i2} \; t_i ],
\label{eq:zhang1}
\end{equation}
где:
\begin{itemize}
\item $K$ — матрица внутренних параметров камеры;
\item $r_{i1}, r_{i2}$ — первые два столбца матрицы вращения $R_i$;
\item $t_i$ — вектор переноса для $i$-го изображения.
\end{itemize}

Поскольку столбцы матрицы вращения ортонормированы, выполняются условия
\begin{equation}
r_{i1}^\top r_{i2} = 0,
\quad
\| r_{i1} \| = \| r_{i2} \|.
\label{eq:orth}
\end{equation}

Эти условия не зависят от внешних параметров и могут быть переписаны
в виде линейных ограничений на элементы матрицы
\begin{equation}
B = K^{-T} K^{-1}.
\label{eq:B}
\end{equation}


Для каждой гомографии формируются два уравнения:
\begin{equation}
v_{12}^\top b = 0,
\quad
(v_{11} - v_{22})^\top b = 0.
\label{eq:zhang_linear}
\end{equation}

Объединяя ограничения для всех изображений, получается система
\begin{equation}
V b = 0,
\label{eq:Vb}
\end{equation}
которая решается методом SVD. После восстановления матрицы $B$ матрица внутренних параметров $K$ вычисляется аналитически.

Заключительным этапом является нелинейная оптимизация параметров камеры путём минимизации ошибки репроекции~\eqref{eq:reprojection}.

\textbf{Алгоритм метода Чжана:}
\begin{quote}
\ttfamily
1. Для каждого изображения шаблона: \\
	\quad 1.1. вычислить гомографию $H_i$ методом DLT; \\
	\quad 1.1. нормализовать $H_i$. \\
2. Для каждой гомографии сформировать линейные ограничения
   на элементы матрицы $B = K^{-T} K^{-1}$. \\
3. Объединить все ограничения в линейную систему $V b = 0$. \\
4. Найти решение системы с помощью SVD. \\
5. Восстановить матрицу внутренних параметров $K$
   из найденной матрицы $B$. \\
6. Выполнить нелинейную оптимизацию параметров камеры
   путём минимизации ошибки репроекции.
\end{quote}

\begin{figure}[H]
  \centering
  \includegraphics[scale=0.5]{images/Result.png}
  \caption{Результат работы алгоритма выравнивания.}
  \label{fig:result}
\end{figure}

\bigskip

\subsection*{Рекомендуемые источники}

\begin{itemize}
\item Camera Calibration and 3D Reconstruction 

\url{https://docs.opencv.org/4.x/d9/d0c/group__calib3d.html}


\item OpenCV Camera Calibration Documentation  

\url{https://docs.opencv.org/4.x/dc/dbb/tutorial_py_calibration.html}


\item Camera Parameters

\url{https://www.youtube.com/watch?v=uHApDqH-8UE}


\item DLT Method

\url{https://www.ipb.uni-bonn.de/html/teaching/msr2-2020/sse2-13-DLT.pdf}
\linebreak
\url{https://www.youtube.com/watch?v=3NcQbZu6xt8}


\item Zhang’s Method

\url{https://www.youtube.com/watch?v=-9He7Nu3u8s}
\linebreak
\url{https://sangillee.com/2025-07-27-calibrating-cameras-zhang-method/}


\item Z. Zhang, “A Flexible New Technique for Camera Calibration”, 1999.


\item GitHub с примерами
\url{https://github.com/57459N/ARnVRLabs}
\end{itemize}

\end{document}
