\documentclass[12pt,a4paper]{article}

\usepackage[utf8]{inputenc}
\usepackage[T2A]{fontenc}
\usepackage[russian]{babel}

\usepackage{amsmath, amssymb}
\usepackage{graphicx}
\usepackage{geometry}
\usepackage{hyperref}

\geometry{margin=2.5cm}

\setlength{\parindent}{0pt}
\setlength{\parskip}{0.8em}

\begin{document}

\section*{Лабораторная работа: калибровка камеры}

\subsection*{Введение}

Калибровка камеры является одной из базовых задач компьютерного зрения и необходима для установления количественной связи между трёхмерной геометрией сцены и её двумерным изображением. Результатом калибровки является набор параметров, описывающих внутреннее устройство камеры, её положение в пространстве, а также оптические искажения.

В данной лабораторной работе рассматривается классический подход к калибровке камеры по плоскому шаблону. В первой части студент использует готовые инструменты библиотеки OpenCV. Во второй части требуется самостоятельно реализовать линейные этапы метода Чжана (Zhang’s method), начиная с оценки гомографий методом DLT (Direct Linear Transform) и заканчивая восстановлением матрицы внутренних параметров камеры.

Цель работы — понять геометрическую модель камеры, физический смысл параметров калибровки и связь между теоретическими формулами и практической реализацией алгоритмов.

\subsection*{Геометрическая модель камеры}

В рамках данной работы используется классическая pinhole-модель камеры. Проекция трёхмерной точки сцены в изображение описывается уравнением
\[
\lambda
\begin{bmatrix}
u \\ v \\ 1
\end{bmatrix}
=
K
\left[
R \mid t
\right]
\begin{bmatrix}
X \\ Y \\ Z \\ 1
\end{bmatrix},
\]
где $(X,Y,Z)$ — координаты точки в мировой системе координат, $(u,v)$ — координаты пикселя на изображении, $\lambda$ — масштабный коэффициент.

Матрица внутренних параметров камеры имеет вид
\[
K =
\begin{bmatrix}
f_x & 0 & c_x \\
0 & f_y & c_y \\
0 & 0 & 1
\end{bmatrix}.
\]
Здесь $f_x$ и $f_y$ — фокусные расстояния в пиксельных координатах, $c_x, c_y$ — координаты главной точки изображения. Параметр сдвига осей (skew) в данной работе считается равным нулю.

Матрица $R \in SO(3)$ и вектор $t \in \mathbb{R}^3$ описывают ориентацию и положение камеры относительно мировой системы координат. В библиотеке OpenCV вращение часто представляется в виде вектора Родрига $rvec$, который однозначно соответствует матрице $R$.

\subsection*{Модель искажений}

Реальные камеры не удовлетворяют идеальной pinhole-модели из-за оптических искажений. Наиболее часто используется следующая модель дисторсии:
\begin{itemize}
\item радиальная дисторсия с коэффициентами $k_1, k_2, k_3$,
\item тангенциальная дисторсия с коэффициентами $p_1, p_2$.
\end{itemize}

Радиальная дисторсия приводит к «бочкообразным» или «подушкообразным» искажениям изображения, особенно заметным на краях кадра. Тангенциальная дисторсия связана с неточным выравниванием оптических элементов.

В рамках данной лабораторной работы допускается использование как полной модели дисторсии, так и её упрощённых вариантов, однако физический смысл всех параметров должен быть понят студенту.

\bigskip
\textbf{Место для иллюстрации:} схема pinhole-модели и пример радиальной дисторсии.

\subsection*{Калибровка камеры средствами OpenCV}

Библиотека OpenCV предоставляет готовую реализацию калибровки камеры по набору соответствий между трёхмерными точками шаблона и их двумерными проекциями на изображении.

На вход алгоритма подаются:
\begin{itemize}
\item координаты точек шаблона в системе координат шаблона;
\item соответствующие им координаты пикселей на изображениях.
\end{itemize}

Результатом являются матрица внутренних параметров $K$, коэффициенты дисторсии, а также набор внешних параметров $(R_i, t_i)$ для каждого изображения.

Средняя ошибка репроекции определяется как
\[
e = \frac{1}{N} \sum_{i=1}^{N} \| x_i - \pi(K, R, t, X_i) \|,
\]
где $\pi(\cdot)$ — оператор проекции. Ошибка измеряется в пикселях и служит основной метрикой качества калибровки.

Типичные значения средней ошибки репроекции для бытовых камер находятся в пределах порядка единиц пикселя и зависят от качества съёмки, количества изображений и модели дисторсии.

\bigskip
\textbf{Место для иллюстрации:} детекция углов шаблона и визуализация ошибки репроекции.

\subsection*{Оценка гомографии методом DLT}

Если все точки шаблона лежат в одной плоскости, их проекция может быть описана с помощью гомографии:
\[
\lambda
\begin{bmatrix}
u \\ v \\ 1
\end{bmatrix}
=
H
\begin{bmatrix}
X \\ Y \\ 1
\end{bmatrix},
\]
где $H$ — матрица гомографии размера $3 \times 3$, определяемая с точностью до масштаба.

Метод Direct Linear Transform (DLT) позволяет оценить гомографию по набору соответствий между точками плоскости и их изображениями. Для каждой пары соответствующих точек формируется линейная система вида
\[
A h = 0,
\]
где $h$ — вектор, содержащий элементы матрицы $H$.

Решение этой системы находится как правый сингулярный вектор матрицы $A$, соответствующий наименьшему сингулярному значению. Для повышения численной устойчивости рекомендуется нормализовать координаты точек перед построением системы.

\textbf{Псевдокод DLT:}
\begin{verbatim}
Normalize 2D and planar points
Build matrix A from point correspondences
Solve Ah = 0 using SVD
Reshape h into matrix H
Denormalize H
\end{verbatim}

\bigskip
\textbf{Место для иллюстрации:} визуализация соответствий и действия гомографии.

\subsection*{Метод Чжана}

Метод Чжана использует тот факт, что для плоского шаблона гомография связана с параметрами камеры следующим образом:
\[
H = K [ r_1 \; r_2 \; t ],
\]
где $r_1$ и $r_2$ — первые два столбца матрицы вращения.

Из ортонормальности столбцов матрицы вращения следуют ограничения:
\[
r_1^\top r_2 = 0, \quad \|r_1\| = \|r_2\|.
\]

Эти условия приводят к линейным ограничениям на элементы матрицы
\[
B = K^{-T} K^{-1},
\]
которые можно записать в виде системы
\[
V b = 0.
\]

Используя гомографии, полученные с разных изображений, формируется переопределённая линейная система, решение которой находится методом SVD. После восстановления матрицы $B$ из неё вычисляется матрица внутренних параметров $K$.

На заключительном этапе проводится нелинейная оптимизация параметров камеры путём минимизации ошибки репроекции. Для этого допускается использование готовых численных солверов.

\textbf{Псевдокод метода Чжана:}
\begin{verbatim}
For each image:
    estimate homography H_i using DLT

Build linear system V from all H_i
Solve Vb = 0 using SVD
Recover intrinsic matrix K
Refine K, R, t and distortion via non-linear optimization
\end{verbatim}

\bigskip
\textbf{Место для иллюстрации:} результат устранения дисторсии до и после оптимизации.

\subsection*{Применение параметров камеры}

Полученные параметры калибровки позволяют устранять искажения изображения и использовать камеру в задачах трёхмерной реконструкции, навигации и визуального трекинга.

Для видеопотока параметры калибровки применяются путём предварительного вычисления карт преобразования и последующего отображения каждого кадра.

\textbf{Псевдокод:}
\begin{verbatim}
Compute undistortion maps
For each frame:
    apply remap
\end{verbatim}

\bigskip
\textbf{Место для иллюстрации:} видеопоток до и после коррекции искажений.

\subsection*{Задания}

\begin{itemize}
\item Ознакомиться с теоретической частью лабораторной работы.
\item Самостоятельно собрать набор изображений калибровочного шаблона.
\item Выполнить калибровку камеры средствами OpenCV и вычислить ошибку репроекции.
\item Использовать полученные параметры для коррекции изображения в видеопотоке в реальном времени.
\item Реализовать оценку гомографии методом DLT.
\item Реализовать линейную часть метода Чжана и выполнить нелинейную оптимизацию параметров.
\item Проанализировать полученные метрики и подготовить скриншоты промежуточных этапов работы.
\item Продемонстрировать результаты и понимание метода преподавателю.
\end{itemize}

\subsection*{Контрольные вопросы и проверка}

При защите лабораторной работы студент должен уметь:
\begin{itemize}
\item объяснить физический смысл параметров $K$, $R$, $t$;
\item интерпретировать коэффициенты дисторсии;
\item описать, как вычисляется ошибка репроекции;
\item объяснить различие между реализацией OpenCV и собственной реализацией метода Чжана;
\item подтвердить использование собственных данных.
\end{itemize}

\subsection*{Литература}

\begin{itemize}
\item OpenCV Camera Calibration Documentation:
\url{https://docs.opencv.org/4.x/dc/dbb/tutorial_py_calibration.html}
\item Camera parameters (video):
\url{https://www.youtube.com/watch?v=uHApDqH-8UE}
\item DLT method (video):
\url{https://www.youtube.com/watch?v=3NcQbZu6xt8}
\item Zhang’s Method (video):
\url{https://www.youtube.com/watch?v=-9He7Nu3u8s}
\item Zhang’s Method (article):
\url{https://sangillee.com/2025-07-27-calibrating-cameras-zhang-method/}
\item Z. Zhang, “A Flexible New Technique for Camera Calibration”, 1999.
\end{itemize}

\end{document}
